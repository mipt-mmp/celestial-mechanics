\documentclass{beamer}

%%% Работа с русским языком

\usepackage{cmap}					% поиск в PDF
\usepackage{mathtext} 				% русские буквы в формулах
\usepackage[T2A]{fontenc}			% кодировка
\usepackage[utf8]{inputenc}			% кодировка исходного текста
\usepackage[english,russian]{babel}	% локализация и переносы

%%% Дополнительная работа с математикой
\usepackage{amsmath,amsfonts,amssymb,amsthm,mathtools} % пакеты AMS

\usepackage{listings}

\title{Прогресс по модели спутника}

\usetheme{Berlin}

\begin{document}
\begin{frame}
  \titlepage
\end{frame}

\section{Что делали}

\begin{frame}
    Из-за закона сохранению импульса раньше улетали, если скорость спутника вначале была ненулевая, сделали точкой отсчёта центр масс $\frac{\sum \vec{r_i} m_i}{\sum m_i}$ со скоростью $\frac{\sum \vec{v_i} m_i}{\sum m_i}$. Теперь мы можем наблюдать за стабильной системой, которая не улетит за экран!
\end{frame}

\begin{frame}
    В качестве дополнительной проверки был добавлен подсчёт закона сохранения момента импульса $\sum \vec{r_i} \times \vec{m_i}$. Получили, что он ведёт себя ещё стабильнее закона сохранения энергии.
\end{frame}

 \begin{frame}
    Для динамического выбора шага дифференцирования предположили, что $dx \sim \min_i \min_j \frac{ \vec{r_{ij}}}{ \vec{v_{ij}}}$. Константа для динамического шага была подобрана эмпирическим путём
 \end{frame}

\begin{frame}
    У нас есть два типа пересчёта состояния системы:

    \begin{itemize}
        \item Сначала скорость, потом расстояние, т.е. $v = a \cdot dx;\; r = v \cdot dx$
        \item Сначала расстояние, потом скорость, т.е. к $r$ сразу прибавляем $a\frac{dx^2}{2}$
    \end{itemize}

    В первом случае энергия падает, во втором растёт.
\end{frame}

\begin{frame}
    Были добавлены флуктуации - прилетает метеорит, который даёт случайный импульс, который на 4-5 порядков ниже импульсов тел, в которые прилетает. Но даже этого хватает, чтобы стабильность была потеряна.
\end{frame}

 \begin{frame}
    Итоги по различным системам

    \begin{itemize}
        \item<1-> Система двух тел (Земля, Солнце) без метеоритов стабильна.
        \item<2-> Система трёх примерно одинаковых тел с метеоритами разлетается.
        \item<3-> Система трёх тел (Земля, Солнце, Луна) с метеоритами стабильна.
    \end{itemize}
 \end{frame}

\end{document}
