\documentclass{beamer}

%%% Работа с русским языком

\usepackage{cmap}					% поиск в PDF
\usepackage{mathtext} 				% русские буквы в формулах
\usepackage[T2A]{fontenc}			% кодировка
\usepackage[utf8]{inputenc}			% кодировка исходного текста
\usepackage[english,russian]{babel}	% локализация и переносы

%%% Дополнительная работа с математикой
\usepackage{amsmath,amsfonts,amssymb,amsthm,mathtools} % пакеты AMS

\usepackage{listings}

\title{Прогресс по модели спутника}

\usetheme{Berlin}

\begin{document}
\begin{frame}
  \titlepage
\end{frame}

\section{Что делали}

\begin{frame}[fragile]
  Для того, чтобы не налажать с вычислениями было принято решение проводить все вычисления через класс-обёртку, ограничения на операции с которым просто не дадут скомпилировать программу.

  \begin{lstlisting}
    template<typename Num, int Meter = 0, 
        int Sec = 0, int KG = 0, int K = 0, 
        int Mole = 0, int Ampere = 0, 
        int Candela = 0>
    class Unit {
      ...
    }
  \end{lstlisting}
\end{frame}

\begin{frame}[fragile]
  Основное действующее лицо - MaterialPoint

  \begin{lstlisting}
    class MaterialPoint {
    // State characteristics
        Position pos_;
        Velocity v_;
        Mass m_;

    // Calculation only;
        Force cur_f_;
        ...
      }
  \end{lstlisting}
\end{frame}

\begin{frame}[fragile]
  Инициализируем вселенную\dots

  \begin{lstlisting}
m_earth = new Celestial(5.9e24_kg, 6.3e6_m, this);
m_earth->setVelocity({0_m / 1_sec, 0_m / 1_sec});

m_earth->setColor(Qt::green);
m_universe.addMaterialPoint(m_earth->getObject());

m_moon = new Celestial(7.36e20_kg, 1.7e6_m, this);
m_moon->setVelocity({0_m / 1_sec, 1.2e3_m / 1_sec});
m_moon->setPosition({4e8_m, 0_m});

m_moon->setColor(Qt::blue);
m_universe.addMaterialPoint(m_moon->getObject());
  \end{lstlisting}
\end{frame}

\begin{frame}[fragile]
  Пересчитываем силы, действующие на тела, в бесконечном цикле, после чего считаем сдвиг всех тел на \textbf{HARDCODED} $dt$.

  \begin{lstlisting}
auto direction = 
-Normalize(m_mps[i]->getPos() - m_mps[j]->getPos());

auto dist = 
(m_mps[i]->getPos() - m_mps[j]->getPos()).Len2();

Force f = 
direction * phys::consts::G * m_mps[i]->getMass() 
          * m_mps[j]->getMass() / dist;

m_mps[i]->applyForce(f);
  \end{lstlisting}
\end{frame}

\section{Какие проблемы}

\begin{frame}
  \begin{itemize}
    \item<1-> Проблемы с масштабом. $R$ планеты $\ll$ расстояния между ними
    \item<2-> Для стабилизации системы приходится подбирать шаг $dt$ и масштаб отрисовки
    \item<3-> Суммарный импульс не сохраняется, картинка уезжает
    \item<4-> Суммарная энергия системы колеблется, поэтому проскакивают ускорения / замедления
  \end{itemize}
\end{frame}


\end{document}