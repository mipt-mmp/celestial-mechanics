\documentclass{beamer}

\usepackage{cmap}
\usepackage{mathtext}
\usepackage[T2A]{fontenc}
\usepackage[utf8]{inputenc}
\usepackage[english,russian]{babel}
\usepackage{amsmath,amsfonts,amssymb,amsthm,mathtools}
\usepackage{listings}

\title{Celestial Mechanics: Final Recap}

\usetheme{Berlin}

\begin{document}

\begin{frame}
  \titlepage
\end{frame}

\section{Recap}

\begin{frame}[fragile]{Problem}
  Создать систему из $2$ (или $n$) тел, смоделировать движение этих тел и действие их друг на друга при помощи закона всемирного тяготения.
\end{frame}

\begin{frame}[fragile]{Solution}
  Чтобы смоделировать движение тел, мы воспольлзовались формулой зависимости координаты тела от времени
  \[\overrightarrow{x} = \overrightarrow{x_0} + \overrightarrow{v} \cdot t + \dfrac{\overrightarrow{a} \cdot t^2}{2}\]
  Или же в нашем случае\newline
  \[d\overrightarrow{x} = \overrightarrow{v} \cdot dt + \dfrac{\overrightarrow{a} \cdot (dt)^2}{2}\]
  Где ускорение вычисляется по формуле $\overrightarrow{a} = \dfrac{\overrightarrow{F}}{m}$
\end{frame}

\begin{frame}[fragile]{Solution}
  Чтобы смоделировать действие тел друг на друга, для каждой пары тел $i, j$ мы применили силу гравитационного притяжения,\newline
  полученную по формуле
  \[\overrightarrow{F_{i,j}} = G \cdot \dfrac{m_i \cdot m_j}{\overrightarrow{r_{i,j}^2}}\]
  Note: Чтобы получить правильное направление действия силы, можно доможнить ее на нормированный вектор\newline
  направления: $-(\overrightarrow{x_i} - \overrightarrow{x_j})$
\end{frame}

\begin{frame}[fragile]{Problem}
  Описанная ранее модель имела существенный недостаток: центр масс системы тел не совпадал с началом нашей системы\newline
  отсчета, в связи с чем вся система пропадала с экрана спустя какое-то время.
\end{frame}

\begin{frame}[fragile]{Solution}
  Решением проблемы стало, очевидно, перемещение начала системы отсчета в центр масс системы тел,\newline
  который вычисляется следующим образом:
  \[\overrightarrow{x_c} = \dfrac{\sum_{i = 1}^{n} m_i \cdot \overrightarrow{x_i}}{\sum_{i = 1}^{n} m_i}\]
  Note: Таким образом, координаты тела в новой системе отсчета получаются так: $x_{new} = x_{orig} - x_c$
\end{frame}

\begin{frame}[fragile]{Problem}
  При перемещении начала сисетмы отсчета в другое место, не сохраняются начальные (заданные пользователем) скорости тел,
  так как это тоже вектора и в новой системе отсчета они будут иметь другие значения.
\end{frame}

\begin{frame}[fragile]{Solution}
  Решение - так же, как и координаты тел, переместить их вектора скоростей с поправкой на мгновенный центр скоростей:
  \[\overrightarrow{v_c} = \dfrac{\sum_{i = 1}^{n} m_i \cdot \overrightarrow{v_i}}{\sum_{i = 1}^{n} m_i}\]
  Note: Таким образом, скорость тела в новой системе отсчета получаются так: $v_{new} = v_{orig} - v_c$
\end{frame}

\begin{frame}[fragile]{Problem}
  Как мы вообще узнаем, корректна ли наша модель?
\end{frame}

\begin{frame}[fragile]{Solution}
  Будем собирать метрики:
  \begin{itemize}
    \item<1-> Общую энергию системы (ведь мы знаем, что в изолированной системе работает ЗСЭ, поэтому
    малость или отсутствие колебаний подтверждали бы корректность)
    \item<2-> Сумму импульсов (все сказанное выше верно и для ЗСИ)
    \item<3-> Сумму моментов импульсов (в отличие от суммы импульсов также дает информацию о том,
    сохранятются ли расстояния и траектории)
  \end{itemize}
\end{frame}

\begin{frame}[fragile]{Results}
  Колебания:
  \begin{itemize}
    \item<1-> Увеличение на $\approx 0.4 \cdot {10}^6 Дж$ в год
    \item<2-> Суммы импульсов в $\pm 7 \cdot {10}^6 \dfrac{м \cdot кг}{с}$
    \item<3-> Увеличение на $\approx 0.1 \cdot {10}^{12} \dfrac{м^2 \cdot кг}{с}$
  \end{itemize}
  \pause
  \pause
  \pause Дают понять, что модель почти наверное корректная :-)\newline
  \pause Note: Запомним эти результаты, мы к ним еще вернемся
\end{frame}

\begin{frame}[fragile]{Solution}
  Решение - так же, как и координаты тел, переместить их вектора скоростей с поправкой на мгновенный центр скоростей:
  \[\overrightarrow{v_c} = \dfrac{\sum_{i = 1}^{n} m_i \cdot \overrightarrow{v_i}}{\sum_{i = 1}^{n} m_i}\]
  Note: Таким образом, скорость тела в новой системе отсчета получаются так: $v_{new} = v_{orig} - v_c$
\end{frame}

\begin{frame}[fragile]{Problem}
  Мы выяснили, что модель $\pm$ корректна, но как мы узнаем устойчива ли она?
\end{frame}

\begin{frame}[fragile]{Solution}
  Раз в секунду (или другое время) к слкчайному телу в системе будем добавлять случанйные флунктуации
  (и назовем их метеоритами), посмотрим, разлетится ли система.\newline
  Спойлер: система не разлетелась, значит она вернее всего устойчива
\end{frame}

\begin{frame}[fragile]{Problem}
  Отлично! С корректной и устойчивой моделью очень удобно работать. За исключением маленбкого ньюанса:
  $dt$ в нашей модели не может меняться динамический, подстраиваясб под систему и делая ее более правдоподобной.
\end{frame}

\begin{frame}[fragile]{Solution}
  Будем вычислять $dt$ динамически, основываясь на следующем соотношении:
  \[dt \sim \min_i \min_j \dfrac{\overrightarrow{r_{i,j}}}{\overrightarrow{v_{i,j}}}\]
  Действительно, было бы удобно, если бы при увеличении скорости, $dt$ бы уменьшался,
  ведь тогда мы будем делать более мелуие шаги, и ошибка будет накапливаться значительно медленнее.\newline
  Прямая пропорциональность расстоянию дает нам возможность аккуратнее все пересчитывать когда тела
  находятся близко друг к другу, что тоже может оказаться полезным.
\end{frame}

\begin{frame}[fragile]{Solution}
  Также дополнительно ограничим $dt$ сверху и снизу, чтобы не было пролагиваний при отрисвоке
  из-за слишком долгих расчетов, и домножим на магическую константу, подобранную эмпирическим
  путем для более красивых и точных результатов :-)
\end{frame}

\end{document}
