\documentclass{beamer}

\usepackage{cmap}
\usepackage{mathtext}
\usepackage[T2A]{fontenc}
\usepackage[utf8]{inputenc}
\usepackage[english,russian]{babel}
\usepackage{amsmath,amsfonts,amssymb,amsthm,mathtools}
\usepackage{listings}

\title{Celestial Mechanics: Final Recap}

\usetheme{Berlin}

\begin{document}

\begin{frame}
  \titlepage
\end{frame}

\section{Recap}

\begin{frame}[fragile]{Problem}
  Создать систему из $2$ (или $n$) тел, смоделировать движение этих тел и действие их друг на друга при помощи закона всемирного тяготения.
\end{frame}

\begin{frame}[fragile]{Solution}
  Чтобы смоделировать движение тел, мы воспольлзовались формулой зависимости координаты тела от времени
  \[\overrightarrow{x} = \overrightarrow{x_0} + \overrightarrow{v} \cdot t + \dfrac{\overrightarrow{a} \cdot t^2}{2}\]
  Или же в нашем случае\newline
  \[d\overrightarrow{x} = \overrightarrow{v} \cdot dt + \dfrac{\overrightarrow{a} \cdot (dt)^2}{2}\]
  Где ускорение вычисляется по формуле $\overrightarrow{a} = \dfrac{\overrightarrow{F}}{m}$
\end{frame}

\begin{frame}[fragile]{Solution}
  Чтобы смоделировать действие тел друг на друга, для каждой пары тел $i, j$ мы применили силу гравитационного притяжения,\newline
  полученную по формуле
  \[\overrightarrow{F_{i,j}} = G \cdot \dfrac{m_i \cdot m_j}{\overrightarrow{r_{i,j}^2}}\]
  Note: Чтобы получить правильное направление действия силы, можно доможнить ее на нормированный вектор\newline
  направления: $-(\overrightarrow{x_i} - \overrightarrow{x_j})$
\end{frame}

\begin{frame}[fragile]{Problem}
  Описанная ранее модель имела существенный недостаток: центр масс системы тел не совпадал с началом нашей системы\newline
  отсчета, в связи с чем вся система пропадала с экрана спустя какое-то время.
\end{frame}

\begin{frame}[fragile]{Solution}
  Решением проблемы стало, очевидно, перемещение начала системы отсчета в центр масс системы тел,\newline
  который вычисляется следующим образом:
  \[\overrightarrow{x_c} = \dfrac{\sum_{i = 1}^{n} m_i \cdot \overrightarrow{x_i}}{\sum_{i = 1}^{n} m_i}\]
  Note: Таким образом, координаты тела в новой системе отсчета получаются так: $x_{new} = x_{orig} - x_c$
\end{frame}

\begin{frame}[fragile]{Problem}
  При перемещении начала сисетмы отсчета в другое место, не сохраняются начальные (заданные пользователем) скорости тел,
  так как это тоже вектора и в новой системе отсчета они будут иметь другие значения.
\end{frame}

\begin{frame}[fragile]{Solution}
  Решение - так же, как и координаты тел, переместить их вектора скоростей с поправкой на мгновенный центр скоростей:
  \[\overrightarrow{v_c} = \dfrac{\sum_{i = 1}^{n} m_i \cdot \overrightarrow{v_i}}{\sum_{i = 1}^{n} m_i}\]
  Note: Таким образом, скорость тела в новой системе отсчета получаются так: $v_{new} = v_{orig} - v_c$
\end{frame}

\begin{frame}[fragile]{Problem}
  Как мы вообще узнаем, корректна ли наша модель?
\end{frame}

\begin{frame}[fragile]{Solution}
  Будем собирать метрики:
  \begin{itemize}
    \item<2-> Общую энергию системы (ведь мы знаем, что в изолированной системе работает ЗСЭ, поэтому
    малость или отсутствие колебаний подтверждали бы корректность)
    \item<3-> Сумму импульсов (все сказанное выше верно и для ЗСИ)
    \item<4-> Сумму моментов импульсов (в отличие от суммы импульсов также дает информацию о том,
    сохранятются ли расстояния и траектории)
  \end{itemize}
  \pause[5]
  Это, конечно, не достаточные условия корректности модели, но необходимые.
  Другими необходимыми условиями являются, например, съезжает ли со временем центр масс, или
  крутиться ли орбита. Но эти условия можно проверить путем простого наблюдения за системой. 
\end{frame}

\begin{frame}[fragile]{Results}
  Колебания:
  \begin{itemize}
    \item<1-> Энергии в $\approx 1,3\%$ в год
    \item<2-> Суммы импульсов в $\pm 7 \cdot {10}^6 \dfrac{м \cdot кг}{с}$
    \item<3-> Суммы моментов импульсов в $\approx 1\%$
  \end{itemize}
  \pause[4]
  Дают понять, что модель почти наверное корректная :-)\newline
  Note: Запомним эти результаты, мы к ним еще вернемся
\end{frame}

\begin{frame}[fragile]{Problem}
  Мы выяснили, что модель $\pm$ корректна, но было бы неплохо узнать, устойчива ли она и когда.
\end{frame}

\begin{frame}[fragile]{Solution}
  Время от времени к случайному телу в системе будем добавлять случанйные флуктуации
  (и назовем их метеоритами), посмотрим, разлетится ли система.\newline
  Спойлер: системы (Земля, Солнце) и (Земля, Солнце, Луна), как и предаолагалось отсаются устойчивыми.
  Система из трех тел в общем случае, как и предполагалось, разлетается
\end{frame}

\begin{frame}[fragile]{Problem}
  Отлично! С корректной и устойчивой моделью очень удобно работать. За исключением маленького ньюанса:
  $dt$ в нашей модели не может меняться динамический, подстраиваясь под систему.
\end{frame}

\begin{frame}[fragile]{Solution}
  Будем вычислять $dt$ динамически, основываясь на следующем соотношении:
  \[dt \sim \min_i \min_j \dfrac{\overrightarrow{r_{i,j}}}{\overrightarrow{v_{i,j}}}\]
  Действительно, было бы удобно, если бы при увеличении скорости, $dt$ бы уменьшался,
  ведь тогда мы будем делать более мелкие шаги, и ошибка будет накапливаться значительно медленнее,
  то есть модель будет более точной.\newline
  Прямая пропорциональность расстоянию дает нам возможность динамически "нормировать" $dt$ для каждой системы,
  а не подбирать для каждой системы нормирующие коэффициенты эмирически.
\end{frame}

\begin{frame}[fragile]{Solution}
  Также дополнительно ограничим $dt$ сверху и снизу, чтобы не было пролагиваний при отрисвоке
  из-за слишком долгих расчетов, и домножим на магическую константу, подобранную эмпирическим
  путем для более красивых и точных результатов :-)
\end{frame}

\section{Updates}

\begin{frame}[fragile]{Problem}
  Вспомним проблему отклонений суммы моментов импульсов и энергии системы. Как ее решить?
\end{frame}

\begin{frame}[fragile]{Solution}
  Решим проблему для системы из двух тел (Земля и Солнце) (так система из $n$ тел, где $n > 2$ в общем случае неустойчива)
  \begin{itemize}
    \item <2-> Для ЗСИ просто добавим ровно столько скорости Земле, чтобы $E_{curr} = E_{prev}$
    \item <3-> Для корректировки суммы моментов импульсов скорость мы поменять не можем, так
    как это изменит $E$. Значит, будем менять траекторию. Аналитически найти мы ее не можем, поэтому
    просто переберем все углы с шагом $6^\circ$, выбрав тот, который дает наименьшее изменение
  \end{itemize}
\end{frame}

\end{document}
